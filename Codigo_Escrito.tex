\documentclass[]{article}
\usepackage{lmodern}
\usepackage{amssymb,amsmath}
\usepackage{ifxetex,ifluatex}
\usepackage{fixltx2e} % provides \textsubscript
\ifnum 0\ifxetex 1\fi\ifluatex 1\fi=0 % if pdftex
  \usepackage[T1]{fontenc}
  \usepackage[utf8]{inputenc}
\else % if luatex or xelatex
  \ifxetex
    \usepackage{mathspec}
  \else
    \usepackage{fontspec}
  \fi
  \defaultfontfeatures{Ligatures=TeX,Scale=MatchLowercase}
\fi
% use upquote if available, for straight quotes in verbatim environments
\IfFileExists{upquote.sty}{\usepackage{upquote}}{}
% use microtype if available
\IfFileExists{microtype.sty}{%
\usepackage{microtype}
\UseMicrotypeSet[protrusion]{basicmath} % disable protrusion for tt fonts
}{}
\usepackage[margin=1in]{geometry}
\usepackage{hyperref}
\hypersetup{unicode=true,
            pdftitle={Restaurantes con Analisis de Colas},
            pdfauthor={Centro de Investigaciones UFM},
            pdfborder={0 0 0},
            breaklinks=true}
\urlstyle{same}  % don't use monospace font for urls
\usepackage{color}
\usepackage{fancyvrb}
\newcommand{\VerbBar}{|}
\newcommand{\VERB}{\Verb[commandchars=\\\{\}]}
\DefineVerbatimEnvironment{Highlighting}{Verbatim}{commandchars=\\\{\}}
% Add ',fontsize=\small' for more characters per line
\usepackage{framed}
\definecolor{shadecolor}{RGB}{248,248,248}
\newenvironment{Shaded}{\begin{snugshade}}{\end{snugshade}}
\newcommand{\AlertTok}[1]{\textcolor[rgb]{0.94,0.16,0.16}{#1}}
\newcommand{\AnnotationTok}[1]{\textcolor[rgb]{0.56,0.35,0.01}{\textbf{\textit{#1}}}}
\newcommand{\AttributeTok}[1]{\textcolor[rgb]{0.77,0.63,0.00}{#1}}
\newcommand{\BaseNTok}[1]{\textcolor[rgb]{0.00,0.00,0.81}{#1}}
\newcommand{\BuiltInTok}[1]{#1}
\newcommand{\CharTok}[1]{\textcolor[rgb]{0.31,0.60,0.02}{#1}}
\newcommand{\CommentTok}[1]{\textcolor[rgb]{0.56,0.35,0.01}{\textit{#1}}}
\newcommand{\CommentVarTok}[1]{\textcolor[rgb]{0.56,0.35,0.01}{\textbf{\textit{#1}}}}
\newcommand{\ConstantTok}[1]{\textcolor[rgb]{0.00,0.00,0.00}{#1}}
\newcommand{\ControlFlowTok}[1]{\textcolor[rgb]{0.13,0.29,0.53}{\textbf{#1}}}
\newcommand{\DataTypeTok}[1]{\textcolor[rgb]{0.13,0.29,0.53}{#1}}
\newcommand{\DecValTok}[1]{\textcolor[rgb]{0.00,0.00,0.81}{#1}}
\newcommand{\DocumentationTok}[1]{\textcolor[rgb]{0.56,0.35,0.01}{\textbf{\textit{#1}}}}
\newcommand{\ErrorTok}[1]{\textcolor[rgb]{0.64,0.00,0.00}{\textbf{#1}}}
\newcommand{\ExtensionTok}[1]{#1}
\newcommand{\FloatTok}[1]{\textcolor[rgb]{0.00,0.00,0.81}{#1}}
\newcommand{\FunctionTok}[1]{\textcolor[rgb]{0.00,0.00,0.00}{#1}}
\newcommand{\ImportTok}[1]{#1}
\newcommand{\InformationTok}[1]{\textcolor[rgb]{0.56,0.35,0.01}{\textbf{\textit{#1}}}}
\newcommand{\KeywordTok}[1]{\textcolor[rgb]{0.13,0.29,0.53}{\textbf{#1}}}
\newcommand{\NormalTok}[1]{#1}
\newcommand{\OperatorTok}[1]{\textcolor[rgb]{0.81,0.36,0.00}{\textbf{#1}}}
\newcommand{\OtherTok}[1]{\textcolor[rgb]{0.56,0.35,0.01}{#1}}
\newcommand{\PreprocessorTok}[1]{\textcolor[rgb]{0.56,0.35,0.01}{\textit{#1}}}
\newcommand{\RegionMarkerTok}[1]{#1}
\newcommand{\SpecialCharTok}[1]{\textcolor[rgb]{0.00,0.00,0.00}{#1}}
\newcommand{\SpecialStringTok}[1]{\textcolor[rgb]{0.31,0.60,0.02}{#1}}
\newcommand{\StringTok}[1]{\textcolor[rgb]{0.31,0.60,0.02}{#1}}
\newcommand{\VariableTok}[1]{\textcolor[rgb]{0.00,0.00,0.00}{#1}}
\newcommand{\VerbatimStringTok}[1]{\textcolor[rgb]{0.31,0.60,0.02}{#1}}
\newcommand{\WarningTok}[1]{\textcolor[rgb]{0.56,0.35,0.01}{\textbf{\textit{#1}}}}
\usepackage{graphicx,grffile}
\makeatletter
\def\maxwidth{\ifdim\Gin@nat@width>\linewidth\linewidth\else\Gin@nat@width\fi}
\def\maxheight{\ifdim\Gin@nat@height>\textheight\textheight\else\Gin@nat@height\fi}
\makeatother
% Scale images if necessary, so that they will not overflow the page
% margins by default, and it is still possible to overwrite the defaults
% using explicit options in \includegraphics[width, height, ...]{}
\setkeys{Gin}{width=\maxwidth,height=\maxheight,keepaspectratio}
\IfFileExists{parskip.sty}{%
\usepackage{parskip}
}{% else
\setlength{\parindent}{0pt}
\setlength{\parskip}{6pt plus 2pt minus 1pt}
}
\setlength{\emergencystretch}{3em}  % prevent overfull lines
\providecommand{\tightlist}{%
  \setlength{\itemsep}{0pt}\setlength{\parskip}{0pt}}
\setcounter{secnumdepth}{0}
% Redefines (sub)paragraphs to behave more like sections
\ifx\paragraph\undefined\else
\let\oldparagraph\paragraph
\renewcommand{\paragraph}[1]{\oldparagraph{#1}\mbox{}}
\fi
\ifx\subparagraph\undefined\else
\let\oldsubparagraph\subparagraph
\renewcommand{\subparagraph}[1]{\oldsubparagraph{#1}\mbox{}}
\fi

%%% Use protect on footnotes to avoid problems with footnotes in titles
\let\rmarkdownfootnote\footnote%
\def\footnote{\protect\rmarkdownfootnote}

%%% Change title format to be more compact
\usepackage{titling}

% Create subtitle command for use in maketitle
\newcommand{\subtitle}[1]{
  \posttitle{
    \begin{center}\large#1\end{center}
    }
}

\setlength{\droptitle}{-2em}

  \title{Restaurantes con Analisis de Colas}
    \pretitle{\vspace{\droptitle}\centering\huge}
  \posttitle{\par}
    \author{Centro de Investigaciones UFM}
    \preauthor{\centering\large\emph}
  \postauthor{\par}
      \predate{\centering\large\emph}
  \postdate{\par}
    \date{11/14/2019}


\begin{document}
\maketitle

Importación de las librerias necesarias para poder correr el codigo

\begin{Shaded}
\begin{Highlighting}[]
\KeywordTok{library}\NormalTok{(readxl)}
\KeywordTok{library}\NormalTok{(dplyr)}
\end{Highlighting}
\end{Shaded}

\begin{verbatim}
## Warning: package 'dplyr' was built under R version 3.5.2
\end{verbatim}

\begin{verbatim}
## 
## Attaching package: 'dplyr'
\end{verbatim}

\begin{verbatim}
## The following objects are masked from 'package:stats':
## 
##     filter, lag
\end{verbatim}

\begin{verbatim}
## The following objects are masked from 'package:base':
## 
##     intersect, setdiff, setequal, union
\end{verbatim}

\begin{Shaded}
\begin{Highlighting}[]
\KeywordTok{library}\NormalTok{(lubridate)}
\end{Highlighting}
\end{Shaded}

\begin{verbatim}
## 
## Attaching package: 'lubridate'
\end{verbatim}

\begin{verbatim}
## The following object is masked from 'package:base':
## 
##     date
\end{verbatim}

\hypertarget{funcion-para-formatear-variables-de-duracion}{%
\section{Función para formatear variables de
duración}\label{funcion-para-formatear-variables-de-duracion}}

Con esta funcion podemos formatear las variables con duración en minutos
y colocarlo en terminos comprensibles en medida que sea posible.

\begin{Shaded}
\begin{Highlighting}[]
\NormalTok{Formatear <-}\StringTok{ }\ControlFlowTok{function}\NormalTok{(variable) \{}
\NormalTok{  nueva <-}\StringTok{ }\KeywordTok{period}\NormalTok{(}\DataTypeTok{hour=}\NormalTok{(variable}\OperatorTok\DecValTok{60}\NormalTok{), }\DataTypeTok{minute=}\NormalTok{(variable}\OperatorTok\DecValTok{60}\OperatorTok\DecValTok{1}\NormalTok{), }\DataTypeTok{second=}\NormalTok{((variable}\OperatorTok\DecValTok{1}\OperatorTok{*}\DecValTok{60}\NormalTok{)}\OperatorTok\DecValTok{1}\NormalTok{))}
  \KeywordTok{return}\NormalTok{(nueva)}
\NormalTok{\}}
\end{Highlighting}
\end{Shaded}

\hypertarget{funcion-generador-de-cola}{%
\section{Función Generador de Cola}\label{funcion-generador-de-cola}}

La función Generador\_de\_Cola declarada a continuación, es necesario
ingresar una fila o un vector de datos conteniendo los datos necesarios
para generar una cola, además tambien se le ingresa la cantidad de
minutos que se simularan, si no se ingresa una cantidad de minutos, por
defecto se pasa una hora a la función si no hay ningún otro dato.

\begin{Shaded}
\begin{Highlighting}[]
\NormalTok{Generador_de_Cola <-}\StringTok{ }\ControlFlowTok{function}\NormalTok{(my_row, }\DataTypeTok{n=}\DecValTok{60}\NormalTok{) \{}
  \CommentTok{# Extraer datos de la fila}
\NormalTok{  cc <-}\StringTok{ }\NormalTok{my_row[}\StringTok{"CC"}\NormalTok{]}
\NormalTok{  restaurante <-}\StringTok{ }\NormalTok{my_row[}\StringTok{"Restaurante"}\NormalTok{]}
\NormalTok{  t_dia <-}\KeywordTok{as.factor}\NormalTok{(my_row[}\StringTok{"FinDe"}\NormalTok{])}
\NormalTok{  miu <-}\StringTok{ }\KeywordTok{as.numeric}\NormalTok{(my_row[}\StringTok{'Miu'}\NormalTok{])}
\NormalTok{  lambda <-}\StringTok{ }\KeywordTok{as.numeric}\NormalTok{(my_row[}\StringTok{'Lambda'}\NormalTok{])}
  
  \CommentTok{#Inicializacion de parametros}
  \CommentTok{# Este vector consiste de }
  \CommentTok{# Id|Tiempo Entre Llegadas|Llegadas|Servicio|Inicio|Tiempo en Cola|Tiempo en Servicio|Final|Tiempo en Sistema|}
\NormalTok{  ultimo_cliente<-}\StringTok{ }\KeywordTok{c}\NormalTok{(}\DecValTok{0}\NormalTok{,}\DecValTok{0}\NormalTok{,}\DecValTok{0}\NormalTok{,}\DecValTok{0}\NormalTok{,}\DecValTok{0}\NormalTok{,}\DecValTok{0}\NormalTok{,}\DecValTok{0}\NormalTok{,}\DecValTok{0}\NormalTok{)}
\NormalTok{  historia_restaurante <-}\StringTok{ }\KeywordTok{c}\NormalTok{(ultimo_cliente)}
  
  \CommentTok{# Crear clientes por cada minuto de la simulación}
\NormalTok{  clientes_por_minuto <-}\StringTok{ }\KeywordTok{rpois}\NormalTok{(n, lambda)}
\NormalTok{  llegadas <-}\StringTok{ }\KeywordTok{c}\NormalTok{()}
  \ControlFlowTok{for}\NormalTok{ (j }\ControlFlowTok{in} \DecValTok{1}\OperatorTok{:}\KeywordTok{length}\NormalTok{(clientes_por_minuto)) \{}
    \CommentTok{# Se generan horas de llegadas de forma aleatoria entre cada minuto }
\NormalTok{    horas <-}\StringTok{ }\KeywordTok{runif}\NormalTok{(clientes_por_minuto[j], j}\DecValTok{-1}\NormalTok{, j) }\OperatorTok\StringTok{ }\KeywordTok{round}\NormalTok{(}\DataTypeTok{digits=}\DecValTok{2}\NormalTok{) }\OperatorTok\StringTok{ }\KeywordTok{sort}\NormalTok{()}
\NormalTok{    llegadas <-}\StringTok{ }\KeywordTok{c}\NormalTok{(llegadas,horas)}
\NormalTok{  \}}
  
  \CommentTok{# Por cada llegada creada calculo el resto de los datos para completar la simulación, cliente por cliente}
  \ControlFlowTok{for}\NormalTok{(i }\ControlFlowTok{in} \DecValTok{1}\OperatorTok{:}\KeywordTok{length}\NormalTok{(llegadas)) \{}
    \CommentTok{# Calculos realizado}
\NormalTok{    llegada <-}\StringTok{ }\NormalTok{llegadas[i]}
\NormalTok{    e_llegadas <-}\StringTok{ }\NormalTok{llegada}\OperatorTok{-}\NormalTok{ultimo_cliente[}\DecValTok{3}\NormalTok{]}
\NormalTok{    servicio <-}\StringTok{ }\KeywordTok{rexp}\NormalTok{(}\DataTypeTok{n =} \DecValTok{1}\NormalTok{, }\DataTypeTok{rate =}\NormalTok{ miu) }\OperatorTok\StringTok{ }\KeywordTok{round}\NormalTok{(}\DataTypeTok{digits=}\DecValTok{2}\NormalTok{)}
\NormalTok{    inicio <-}\StringTok{ }\KeywordTok{ifelse}\NormalTok{(llegada}\OperatorTok{>}\NormalTok{ultimo_cliente[}\DecValTok{7}\NormalTok{],llegada,ultimo_cliente[}\DecValTok{7}\NormalTok{])}
\NormalTok{    cola <-}\StringTok{ }\NormalTok{(inicio}\OperatorTok{-}\NormalTok{llegada) }\OperatorTok\StringTok{ }\KeywordTok{round}\NormalTok{(}\DecValTok{2}\NormalTok{)}
\NormalTok{    final <-}\StringTok{ }\NormalTok{inicio }\OperatorTok{+}\StringTok{ }\NormalTok{servicio}
\NormalTok{    en_sistema <-}\StringTok{ }\NormalTok{final}\OperatorTok{-}\NormalTok{llegada}
    
    \CommentTok{# Reasigno las variables que van cambiando y agrego la fila a la historia}
\NormalTok{    cliente_nuevo <-}\StringTok{ }\KeywordTok{c}\NormalTok{(i, e_llegadas, llegada, inicio, cola, servicio, final, en_sistema)}
\NormalTok{    historia_restaurante <-}\StringTok{ }\KeywordTok{rbind}\NormalTok{(historia_restaurante, cliente_nuevo)}
\NormalTok{    ultimo_cliente <-}\StringTok{ }\NormalTok{cliente_nuevo}
\NormalTok{  \}}
  
  \CommentTok{# Reformatear la tabla}
\NormalTok{  historia_restaurante <-}\StringTok{ }\NormalTok{historia_restaurante[(}\DecValTok{2}\OperatorTok{:}\KeywordTok{nrow}\NormalTok{(historia_restaurante)),]}
\NormalTok{  historia_restaurante <-}\StringTok{ }\KeywordTok{cbind}\NormalTok{(cc, restaurante, t_dia, historia_restaurante) }
  
  \KeywordTok{colnames}\NormalTok{(historia_restaurante) <-}\StringTok{ }\KeywordTok{c}\NormalTok{(}\StringTok{"CC"}\NormalTok{, }\StringTok{"Restaurante"}\NormalTok{, }\StringTok{"FinDe"}\NormalTok{,}\StringTok{"id"}\NormalTok{, }\StringTok{"Entre_Llegadas"}\NormalTok{, }\StringTok{"Llegada"}\NormalTok{, }\StringTok{"Inicio"}\NormalTok{, }\StringTok{"Cola"}\NormalTok{, }\StringTok{"Servicio"}\NormalTok{, }\StringTok{"Final"}\NormalTok{, }\StringTok{"En_Sistema"}\NormalTok{)}
\NormalTok{  historia_restaurante <-}\StringTok{ }\KeywordTok{as_data_frame}\NormalTok{(historia_restaurante) }
  
  \CommentTok{# Cambiar las variables de tiempo }
\NormalTok{  historia_restaurante<-}\StringTok{ }\NormalTok{historia_restaurante }\OperatorTok
\StringTok{    }\KeywordTok{mutate_at}\NormalTok{(}\KeywordTok{c}\NormalTok{(}\StringTok{"Entre_Llegadas"}\NormalTok{, }\StringTok{"Llegada"}\NormalTok{, }\StringTok{"Inicio"}\NormalTok{, }\StringTok{"Cola"}\NormalTok{, }\StringTok{"Servicio"}\NormalTok{, }\StringTok{"Final"}\NormalTok{, }\StringTok{"En_Sistema"}\NormalTok{), as.numeric)}
\NormalTok{  historia_restaurante<-}\StringTok{ }\NormalTok{historia_restaurante }\OperatorTok
\StringTok{    }\KeywordTok{mutate_at}\NormalTok{(}\KeywordTok{c}\NormalTok{(}\StringTok{"Entre_Llegadas"}\NormalTok{, }\StringTok{"Llegada"}\NormalTok{, }\StringTok{"Inicio"}\NormalTok{, }\StringTok{"Cola"}\NormalTok{, }\StringTok{"Servicio"}\NormalTok{, }\StringTok{"Final"}\NormalTok{, }\StringTok{"En_Sistema"}\NormalTok{), }\OperatorTok{~}\KeywordTok{Formatear}\NormalTok{(.))}

  \KeywordTok{return}\NormalTok{(historia_restaurante)}
\NormalTok{\} }
\end{Highlighting}
\end{Shaded}

\hypertarget{lectura-y-manipulacion-de-datos}{%
\section{Lectura y Manipulación de
Datos}\label{lectura-y-manipulacion-de-datos}}

Este chunk de código contiene tres ciclos anidados (uno dentro del
otro), los cuales son utiles para la lectura de los archivos, no
importando la cantidad de estos. Está estructurado para considerar las
carpetas como centros comerciales, cada archivo debe estar en formato
xlsx, incluir la palabra Datos al inicio y Seguido por el nombre del
restaurante, sin caracteres especiales ni espacios.

Además de leer los archivos con este codigo se crean columnas calculadas
con los tiempos utiles y finalmente se crea una linea con los datos
utiles (resumidos) por cada una de las hojas en todos los archivos. Cada
una de estas filas se adjunta a una tabla llamada datos\_iniciales que
almacenara todos los datos que más adelante serán utilizados. De este
codigo unicamente se recibe una linea por cada una de las hojas de excel
que es leida. Además se almacena la tabla completa de datos iniciales.

\begin{Shaded}
\begin{Highlighting}[]
\CommentTok{## Ingresar a todos los Centros Comerciales}
\ControlFlowTok{for}\NormalTok{ (folder }\ControlFlowTok{in} \KeywordTok{list.files}\NormalTok{(}\StringTok{"Data"}\NormalTok{)) \{}
  \CommentTok{## Ingresar a todos los Restaurantes}
  \ControlFlowTok{for}\NormalTok{ (archivo }\ControlFlowTok{in} \KeywordTok{list.files}\NormalTok{(}\KeywordTok{paste}\NormalTok{(}\StringTok{"Data"}\NormalTok{, folder, }\DataTypeTok{sep =}\StringTok{"/"}\NormalTok{ ))) \{}
    \CommentTok{## Ingresar a todos los dias}
    \ControlFlowTok{for}\NormalTok{ (sheet }\ControlFlowTok{in} \DecValTok{1}\OperatorTok{:}\KeywordTok{length}\NormalTok{(}\KeywordTok{excel_sheets}\NormalTok{(}\KeywordTok{paste}\NormalTok{(}\StringTok{"Data"}\NormalTok{, folder, archivo, }\DataTypeTok{sep =} \StringTok{"/"}\NormalTok{)))) \{}
      \CommentTok{# Solo para poder ver que estoy sacando}
      \KeywordTok{print}\NormalTok{(}\KeywordTok{paste}\NormalTok{(}\StringTok{"Leyendo: Data/"}\NormalTok{, folder,}\StringTok{'/'}\NormalTok{, archivo,}\StringTok{': hoja-'}\NormalTok{,sheet, }\DataTypeTok{sep =} \StringTok{""}\NormalTok{))}
      
      \CommentTok{# Importar tabla de datos}
\NormalTok{      temp_table <-}\StringTok{ }\KeywordTok{read_excel}\NormalTok{(}\DataTypeTok{path =} \KeywordTok{paste}\NormalTok{(}\StringTok{"Data"}\NormalTok{, folder, archivo, }\DataTypeTok{sep =} \StringTok{"/"}\NormalTok{), }\DataTypeTok{sheet =}\NormalTok{ sheet, }\DataTypeTok{skip =} \DecValTok{3}\NormalTok{,}
                               \CommentTok{# Se coloca la esquina superior derecha se colocan los tipos de datos}
                               \DataTypeTok{range =} \KeywordTok{cell_limits}\NormalTok{(}\KeywordTok{c}\NormalTok{(}\DecValTok{4}\NormalTok{, }\OtherTok{NA}\NormalTok{), }\KeywordTok{c}\NormalTok{(}\OtherTok{NA}\NormalTok{, }\DecValTok{4}\NormalTok{)),}\DataTypeTok{col_types =} \KeywordTok{c}\NormalTok{(}\StringTok{"numeric"}\NormalTok{, }\StringTok{"date"}\NormalTok{, }\StringTok{"date"}\NormalTok{, }\StringTok{"date"}\NormalTok{))}
      \CommentTok{# Renombramos las columnas}
      \KeywordTok{colnames}\NormalTok{(temp_table) =}\StringTok{ }\KeywordTok{c}\NormalTok{(}\StringTok{'Ingreso'}\NormalTok{, }\StringTok{'Llegada'}\NormalTok{, }\StringTok{'Inicio'}\NormalTok{, }\StringTok{'Final'}\NormalTok{)}
      
      \CommentTok{# Se dejan unicamente las filas completas pues el codigo anterior lee mas de las que existen y se laguean las llegadas}
\NormalTok{      temp_table<-}\StringTok{ }\NormalTok{temp_table  }\OperatorTok\StringTok{ }\KeywordTok{filter}\NormalTok{(}\KeywordTok{complete.cases}\NormalTok{(temp_table)) }\OperatorTok\StringTok{ }\KeywordTok{mutate}\NormalTok{(}\DataTypeTok{Llegada_Anterior=} \KeywordTok{lag}\NormalTok{(Llegada,}\DecValTok{1}\NormalTok{), }\DataTypeTok{Final_Anterior=} \KeywordTok{lag}\NormalTok{(Final,}\DecValTok{1}\NormalTok{)) }\OperatorTok\StringTok{ }
\StringTok{        }\CommentTok{# Se estiman los diferentes tiempos que se tienen de la toma de datos que son relevantes para las colas.}
\StringTok{        }\KeywordTok{mutate}\NormalTok{(}\DataTypeTok{T_Llegadas=}\KeywordTok{as.duration}\NormalTok{(Llegada}\OperatorTok{-}\NormalTok{Llegada_Anterior), }\DataTypeTok{T_Cola=} \KeywordTok{as.duration}\NormalTok{(Inicio}\OperatorTok{-}\NormalTok{Llegada), }
               \DataTypeTok{T_Servicio=}\KeywordTok{as.duration}\NormalTok{(Final}\OperatorTok{-}\NormalTok{Inicio), }\DataTypeTok{T_Sistema=} \KeywordTok{as.duration}\NormalTok{(Final}\OperatorTok{-}\NormalTok{Llegada),}
               
               \DataTypeTok{Libre=} \KeywordTok{as.duration}\NormalTok{(}\KeywordTok{ifelse}\NormalTok{(Llegada}\OperatorTok{>}\NormalTok{Final_Anterior, Final}\OperatorTok{-}\NormalTok{Llegada,}\DecValTok{0}\NormalTok{))) }\OperatorTok\StringTok{ }
\StringTok{        }\CommentTok{# Transformar a minutos en el Tiempo de Cola se estiman 6 segundos para considerar las mala toma de datos}
\StringTok{        }\KeywordTok{mutate}\NormalTok{(}\DataTypeTok{T_Llegadas=}\NormalTok{T_Llegadas}\OperatorTok{/}\KeywordTok{dminutes}\NormalTok{(}\DecValTok{1}\NormalTok{), }\DataTypeTok{T_Cola=} \KeywordTok{ifelse}\NormalTok{(T_Cola}\OperatorTok{/}\KeywordTok{dminutes}\NormalTok{(}\DecValTok{1}\NormalTok{) }\OperatorTok{<}\StringTok{ }\FloatTok{0.1}\NormalTok{, }\DecValTok{0}\NormalTok{, T_Cola}\OperatorTok{/}\KeywordTok{dminutes}\NormalTok{(}\DecValTok{1}\NormalTok{)), }\DataTypeTok{T_Servicio=}\NormalTok{T_Servicio}\OperatorTok{/}\KeywordTok{dminutes}\NormalTok{(}\DecValTok{1}\NormalTok{), }\DataTypeTok{T_Sistema=}\NormalTok{ T_Sistema}\OperatorTok{/}\KeywordTok{dminutes}\NormalTok{(}\DecValTok{1}\NormalTok{),}
               
               \DataTypeTok{Libre=}\NormalTok{ Libre}\OperatorTok{/}\KeywordTok{dminutes}\NormalTok{(}\DecValTok{1}\NormalTok{), }\DataTypeTok{Hizo_Cola=}\KeywordTok{ifelse}\NormalTok{(T_Cola}\OperatorTok{!=}\DecValTok{0}\NormalTok{, }\DecValTok{1}\NormalTok{, }\DecValTok{0}\NormalTok{))}
      \CommentTok{# Con estos ciclos, calculo cual es el promedio de personas en el sistema y el promedio de personas en Cola}
\NormalTok{      En_Sistema<-}\KeywordTok{c}\NormalTok{(}\DecValTok{0}\NormalTok{)}
\NormalTok{      En_Cola<-}\KeywordTok{c}\NormalTok{(}\DecValTok{0}\NormalTok{)}
      \ControlFlowTok{for}\NormalTok{ (i }\ControlFlowTok{in} \DecValTok{2}\OperatorTok{:}\KeywordTok{nrow}\NormalTok{(temp_table)) \{}
\NormalTok{        contador_s <-}\StringTok{ }\DecValTok{0}
\NormalTok{        contador_c<-}\StringTok{ }\KeywordTok{ifelse}\NormalTok{(temp_table}\OperatorTok{$}\NormalTok{T_Cola[i]}\OperatorTok{>}\DecValTok{0}\NormalTok{,}\DecValTok{1}\NormalTok{,}\DecValTok{0}\NormalTok{)}
        \ControlFlowTok{for}\NormalTok{ (j }\ControlFlowTok{in} \DecValTok{1}\OperatorTok{:}\NormalTok{(i}\DecValTok{-1}\NormalTok{)) \{}
\NormalTok{          contador_s <-}\StringTok{ }\NormalTok{contador_s }\OperatorTok{+}\StringTok{ }\KeywordTok{ifelse}\NormalTok{(temp_table}\OperatorTok{$}\NormalTok{Final[j]}\OperatorTok{>}\NormalTok{temp_table}\OperatorTok{$}\NormalTok{Llegada[i],}\DecValTok{1}\NormalTok{,}\DecValTok{0}\NormalTok{)}
\NormalTok{          contador_c <-}\StringTok{ }\NormalTok{contador_c }\OperatorTok{+}\StringTok{ }\KeywordTok{ifelse}\NormalTok{(temp_table}\OperatorTok{$}\NormalTok{Inicio[j]}\OperatorTok{>}\NormalTok{temp_table}\OperatorTok{$}\NormalTok{Llegada[i],}\DecValTok{1}\NormalTok{,}\DecValTok{0}\NormalTok{)}
\NormalTok{        \}}
\NormalTok{        En_Sistema<-}\StringTok{ }\KeywordTok{c}\NormalTok{(En_Sistema, contador_s)}
\NormalTok{        En_Cola<-}\StringTok{ }\KeywordTok{c}\NormalTok{(En_Cola, contador_c)}
\NormalTok{      \}}
\NormalTok{      temp_table}\OperatorTok{$}\NormalTok{En_Sistema<-}\StringTok{ }\NormalTok{En_Sistema}
\NormalTok{      temp_table}\OperatorTok{$}\NormalTok{En_Cola <-}\StringTok{ }\NormalTok{En_Cola}
      
      
      \CommentTok{# Se obtienen los datos mas relevantes de cada archivo y se resumen en esta tabla}
\NormalTok{      datos_utiles <-}\StringTok{ }\NormalTok{temp_table  }\OperatorTok\StringTok{ }
\StringTok{        }\KeywordTok{summarise}\NormalTok{(}\DataTypeTok{CC=}\NormalTok{ folder, }\DataTypeTok{Restaurante=} \KeywordTok{substr}\NormalTok{(archivo, }\DecValTok{6}\NormalTok{, }\KeywordTok{nchar}\NormalTok{(archivo)}\OperatorTok{-}\DecValTok{5}\NormalTok{), }\DataTypeTok{Fecha=} \KeywordTok{date}\NormalTok{(Llegada[}\DecValTok{1}\NormalTok{]),}\DataTypeTok{Cantidad=} \KeywordTok{n}\NormalTok{(),}
                  \DataTypeTok{Intervalo=}\NormalTok{ (Llegada[}\DecValTok{1}\NormalTok{]}\OperatorTok\NormalTok{Llegada[}\KeywordTok{nrow}\NormalTok{(temp_table)]), }\DataTypeTok{Intervalo2=}\NormalTok{ (Llegada[}\DecValTok{1}\NormalTok{]}\OperatorTok\NormalTok{Final[}\KeywordTok{nrow}\NormalTok{(temp_table)]),}
                  \CommentTok{# Datos para poder realizar simulacion}
                  \DataTypeTok{Inv_Lambda=} \KeywordTok{mean}\NormalTok{(T_Llegadas, }\DataTypeTok{na.rm =}\NormalTok{ T), }\DataTypeTok{Sd_Inv_Lambda =} \KeywordTok{sd}\NormalTok{(T_Llegadas, }\DataTypeTok{na.rm=}\NormalTok{T) , }\DataTypeTok{Miu=} \KeywordTok{mean}\NormalTok{(T_Servicio), }\DataTypeTok{Sd_Miu=} \KeywordTok{sd}\NormalTok{(T_Servicio),}
                  \DataTypeTok{Min_T_Llegadas=}\KeywordTok{min}\NormalTok{(T_Llegadas, }\DataTypeTok{na.rm =}\NormalTok{ T), }\DataTypeTok{Max_T_Llegadas=} \KeywordTok{max}\NormalTok{(T_Llegadas, }\DataTypeTok{na.rm =}\NormalTok{ T), }\DataTypeTok{Max_T_Cola=} \KeywordTok{max}\NormalTok{(T_Cola, }\DataTypeTok{na.rm =}\NormalTok{ T),}
                  \CommentTok{# Promedios de tiempos calculados}
                  \DataTypeTok{T_Cola=} \KeywordTok{mean}\NormalTok{(T_Cola), }\DataTypeTok{T_Sistema_Prom=}\KeywordTok{mean}\NormalTok{(T_Sistema),}
                  \DataTypeTok{T_Sistema_Total=} \KeywordTok{sum}\NormalTok{(T_Sistema), }\DataTypeTok{Libre=}\KeywordTok{sum}\NormalTok{(Libre, }\DataTypeTok{na.rm =}\NormalTok{ T), }\DataTypeTok{Hizo_Cola=} \KeywordTok{sum}\NormalTok{(Hizo_Cola),}
                  \DataTypeTok{En_Cola=} \KeywordTok{mean}\NormalTok{(En_Cola), }\DataTypeTok{En_Sistema=} \KeywordTok{mean}\NormalTok{(En_Sistema)) }\OperatorTok\StringTok{ }
\StringTok{        }\KeywordTok{mutate}\NormalTok{(}\DataTypeTok{D_Semana=}\KeywordTok{wday}\NormalTok{(Fecha,}\DataTypeTok{week_start =} \DecValTok{1}\NormalTok{)) }\OperatorTok\StringTok{ }\KeywordTok{mutate}\NormalTok{(}\DataTypeTok{FinDe =} \KeywordTok{ifelse}\NormalTok{(D_Semana}\OperatorTok{>=}\DecValTok{6}\NormalTok{,}\DecValTok{1}\NormalTok{,}\DecValTok{0}\NormalTok{)) }\OperatorTok\StringTok{ }
\StringTok{        }\KeywordTok{mutate}\NormalTok{(}\DataTypeTok{Tiempo_Tot=} \KeywordTok{as.duration}\NormalTok{(Intervalo)}\OperatorTok{/}\KeywordTok{dhours}\NormalTok{(}\DecValTok{1}\NormalTok{)) }\OperatorTok\StringTok{ }
\StringTok{        }\KeywordTok{mutate}\NormalTok{(}\DataTypeTok{Lambda=}\NormalTok{ Cantidad}\OperatorTok{/}\NormalTok{(Tiempo_Tot}\OperatorTok{*}\DecValTok{60}\NormalTok{)) }\OperatorTok\StringTok{ }
\StringTok{        }\KeywordTok{mutate}\NormalTok{(}\DataTypeTok{Total_Minutos=} \KeywordTok{as.duration}\NormalTok{(Intervalo2)}\OperatorTok{/}\KeywordTok{dminutes}\NormalTok{(}\DecValTok{1}\NormalTok{)) }
      
      \CommentTok{# Si no existe crear tabla datos_iniciales}
      \ControlFlowTok{if}\NormalTok{ (}\KeywordTok{exists}\NormalTok{(}\StringTok{"datos_iniciales"}\NormalTok{)) \{}
\NormalTok{        datos_iniciales <-}\KeywordTok{rbind}\NormalTok{(datos_iniciales, datos_utiles)}
\NormalTok{      \} }\ControlFlowTok{else}\NormalTok{ \{}
\NormalTok{        datos_iniciales <-}\StringTok{ }\NormalTok{datos_utiles}
\NormalTok{      \}}
\NormalTok{    \}}
\NormalTok{  \}}
\NormalTok{\}}
\end{Highlighting}
\end{Shaded}

\begin{verbatim}
## [1] "Leyendo: Data/La Pradera/DatosBurgerKing.xlsx: hoja-1"
## [1] "Leyendo: Data/La Pradera/DatosBurgerKing.xlsx: hoja-2"
## [1] "Leyendo: Data/La Pradera/DatosBurgerKing.xlsx: hoja-3"
## [1] "Leyendo: Data/La Pradera/DatosBurgerKing.xlsx: hoja-4"
## [1] "Leyendo: Data/La Pradera/DatosGoGreen.xlsx: hoja-1"
## [1] "Leyendo: Data/La Pradera/DatosGoGreen.xlsx: hoja-2"
## [1] "Leyendo: Data/La Pradera/DatosGoGreen.xlsx: hoja-3"
## [1] "Leyendo: Data/La Pradera/DatosPandaExpress.xlsx: hoja-1"
## [1] "Leyendo: Data/La Pradera/DatosPandaExpress.xlsx: hoja-2"
## [1] "Leyendo: Data/La Pradera/DatosPandaExpress.xlsx: hoja-3"
## [1] "Leyendo: Data/La Pradera/DatosPolloCampero.xlsx: hoja-1"
## [1] "Leyendo: Data/La Pradera/DatosPolloCampero.xlsx: hoja-2"
## [1] "Leyendo: Data/La Pradera/DatosPolloCampero.xlsx: hoja-3"
## [1] "Leyendo: Data/La Pradera/DatosSubway.xlsx: hoja-1"
## [1] "Leyendo: Data/La Pradera/DatosSubway.xlsx: hoja-2"
## [1] "Leyendo: Data/La Pradera/DatosSubway.xlsx: hoja-3"
## [1] "Leyendo: Data/La Pradera/DatosSubway.xlsx: hoja-4"
## [1] "Leyendo: Data/La Pradera/DatosTacoBell.xlsx: hoja-1"
## [1] "Leyendo: Data/La Pradera/DatosTacoBell.xlsx: hoja-2"
## [1] "Leyendo: Data/La Pradera/DatosTacoBell.xlsx: hoja-3"
## [1] "Leyendo: Data/Miraflores/DatosBurgerKing.xlsx: hoja-1"
## [1] "Leyendo: Data/Miraflores/DatosBurgerKing.xlsx: hoja-2"
## [1] "Leyendo: Data/Miraflores/DatosBurgerKing.xlsx: hoja-3"
## [1] "Leyendo: Data/Miraflores/DatosGoGreen.xlsx: hoja-1"
## [1] "Leyendo: Data/Miraflores/DatosGoGreen.xlsx: hoja-2"
## [1] "Leyendo: Data/Miraflores/DatosGoGreen.xlsx: hoja-3"
## [1] "Leyendo: Data/Miraflores/DatosKFC.xlsx: hoja-1"
## [1] "Leyendo: Data/Miraflores/DatosKFC.xlsx: hoja-2"
## [1] "Leyendo: Data/Miraflores/DatosKFC.xlsx: hoja-3"
## [1] "Leyendo: Data/Miraflores/DatosMcDonalds.xlsx: hoja-1"
## [1] "Leyendo: Data/Miraflores/DatosMcDonalds.xlsx: hoja-2"
## [1] "Leyendo: Data/Miraflores/DatosMcDonalds.xlsx: hoja-3"
## [1] "Leyendo: Data/Miraflores/DatosPolloCampero.xlsx: hoja-1"
## [1] "Leyendo: Data/Miraflores/DatosPolloCampero.xlsx: hoja-2"
## [1] "Leyendo: Data/Miraflores/DatosPolloCampero.xlsx: hoja-3"
## [1] "Leyendo: Data/Miraflores/DatosPolloCampero.xlsx: hoja-4"
## [1] "Leyendo: Data/Miraflores/DatosSubway.xlsx: hoja-1"
## [1] "Leyendo: Data/Miraflores/DatosSubway.xlsx: hoja-2"
## [1] "Leyendo: Data/Miraflores/DatosSubway.xlsx: hoja-3"
## [1] "Leyendo: Data/Miraflores/DatosTacoBell.xlsx: hoja-1"
## [1] "Leyendo: Data/Miraflores/DatosTacoBell.xlsx: hoja-2"
## [1] "Leyendo: Data/Miraflores/DatosTacoBell.xlsx: hoja-3"
## [1] "Leyendo: Data/Pradera Concepción/DatosBurgerKing.xlsx: hoja-1"
## [1] "Leyendo: Data/Pradera Concepción/DatosBurgerKing.xlsx: hoja-2"
## [1] "Leyendo: Data/Pradera Concepción/DatosBurgerKing.xlsx: hoja-3"
## [1] "Leyendo: Data/Pradera Concepción/DatosGoGreen.xlsx: hoja-1"
## [1] "Leyendo: Data/Pradera Concepción/DatosGoGreen.xlsx: hoja-2"
## [1] "Leyendo: Data/Pradera Concepción/DatosGoGreen.xlsx: hoja-3"
## [1] "Leyendo: Data/Pradera Concepción/DatosMcDonalds.xlsx: hoja-1"
## [1] "Leyendo: Data/Pradera Concepción/DatosMcDonalds.xlsx: hoja-2"
## [1] "Leyendo: Data/Pradera Concepción/DatosMcDonalds.xlsx: hoja-3"
## [1] "Leyendo: Data/Pradera Concepción/DatosPandaExpress.xlsx: hoja-1"
## [1] "Leyendo: Data/Pradera Concepción/DatosPandaExpress.xlsx: hoja-2"
## [1] "Leyendo: Data/Pradera Concepción/DatosPandaExpress.xlsx: hoja-3"
## [1] "Leyendo: Data/Pradera Concepción/DatosPolloCampero.xlsx: hoja-1"
## [1] "Leyendo: Data/Pradera Concepción/DatosPolloCampero.xlsx: hoja-2"
## [1] "Leyendo: Data/Pradera Concepción/DatosPolloCampero.xlsx: hoja-3"
## [1] "Leyendo: Data/Pradera Concepción/DatosSubway.xlsx: hoja-1"
## [1] "Leyendo: Data/Pradera Concepción/DatosSubway.xlsx: hoja-2"
## [1] "Leyendo: Data/Pradera Concepción/DatosSubway.xlsx: hoja-3"
## [1] "Leyendo: Data/Pradera Concepción/DatosTacoBell.xlsx: hoja-1"
## [1] "Leyendo: Data/Pradera Concepción/DatosTacoBell.xlsx: hoja-2"
## [1] "Leyendo: Data/Pradera Concepción/DatosTacoBell.xlsx: hoja-3"
\end{verbatim}

\begin{Shaded}
\begin{Highlighting}[]
\NormalTok{datos_a_usar <-}\StringTok{ }\NormalTok{datos_iniciales }\OperatorTok
\StringTok{  }\CommentTok{# S es utilizada para marcar la diferencia de las variables con pesos}
\StringTok{  }\KeywordTok{mutate}\NormalTok{(}\DataTypeTok{LambdaS=}\NormalTok{(Cantidad}\OperatorTok{*}\NormalTok{Lambda), }\DataTypeTok{MiuS=}\NormalTok{(Cantidad}\OperatorTok{*}\NormalTok{Miu), }\DataTypeTok{Var_Miu=}\NormalTok{ (Sd_Miu}\OperatorTok{^}\DecValTok{2}\NormalTok{), }\DataTypeTok{Inv_LambdaS=}\NormalTok{(Cantidad}\OperatorTok{*}\NormalTok{Inv_Lambda), }
         \DataTypeTok{Var_Inv_Lambda=}\NormalTok{(Sd_Inv_Lambda}\OperatorTok{^}\DecValTok{2}\NormalTok{), }\DataTypeTok{En_ColaS=}\NormalTok{ Cantidad}\OperatorTok{*}\NormalTok{En_Cola, }\DataTypeTok{En_SistemaS=}\NormalTok{ Cantidad}\OperatorTok{*}\NormalTok{En_Sistema,}
         \DataTypeTok{T_ColaS=}\NormalTok{ Cantidad}\OperatorTok{*}\NormalTok{T_Cola, }\DataTypeTok{T_Sistema_PromS=}\NormalTok{ Cantidad}\OperatorTok{*}\NormalTok{T_Sistema_Prom) }\OperatorTok\StringTok{ }\KeywordTok{group_by}\NormalTok{(CC, Restaurante, FinDe) }\OperatorTok\StringTok{ }
\StringTok{  }\CommentTok{# Resumen de los datos relevantes por agrupacion establecida}
\StringTok{  }\KeywordTok{summarise}\NormalTok{(}\DataTypeTok{Cantidad =} \KeywordTok{sum}\NormalTok{(Cantidad), }\DataTypeTok{Lambda=} \KeywordTok{sum}\NormalTok{(LambdaS)}\OperatorTok{/}\KeywordTok{sum}\NormalTok{(Cantidad),}
            \DataTypeTok{Miu=} \DecValTok{1}\OperatorTok{/}\NormalTok{(}\KeywordTok{sum}\NormalTok{(MiuS)}\OperatorTok{/}\KeywordTok{sum}\NormalTok{(Cantidad)), }\DataTypeTok{Sd_Miu=}\DecValTok{1}\OperatorTok{/}\NormalTok{(}\KeywordTok{sqrt}\NormalTok{(}\KeywordTok{sum}\NormalTok{(Var_Miu))), }
            \DataTypeTok{Min_T_Llegadas=} \KeywordTok{min}\NormalTok{(Min_T_Llegadas), }\DataTypeTok{Max_T_Llegadas=}\KeywordTok{max}\NormalTok{(Max_T_Llegadas), }\DataTypeTok{Max_T_Cola=} \KeywordTok{max}\NormalTok{(Max_T_Cola),}
            \DataTypeTok{Inv_Lambda=}\KeywordTok{sum}\NormalTok{(Inv_LambdaS)}\OperatorTok{/}\KeywordTok{sum}\NormalTok{(Cantidad), }\DataTypeTok{Sd_Inv_Lambda =} \KeywordTok{sqrt}\NormalTok{(}\KeywordTok{sum}\NormalTok{(Var_Inv_Lambda)),}
            \DataTypeTok{Inv_Miu =}\NormalTok{ (}\KeywordTok{sum}\NormalTok{(MiuS)}\OperatorTok{/}\KeywordTok{sum}\NormalTok{(Cantidad)),}
            \DataTypeTok{T_Cola=} \KeywordTok{sum}\NormalTok{(T_ColaS)}\OperatorTok{/}\KeywordTok{sum}\NormalTok{(Cantidad), }\DataTypeTok{T_Sistema_Prom =} \KeywordTok{sum}\NormalTok{(T_Sistema_PromS)}\OperatorTok{/}\KeywordTok{sum}\NormalTok{(Cantidad),}
            \DataTypeTok{Tiempo_Tot=}\KeywordTok{sum}\NormalTok{(Tiempo_Tot),}
            \CommentTok{# Caracteristicas de operación calculadas}
            \DataTypeTok{Total_Minutos=} \KeywordTok{sum}\NormalTok{(Total_Minutos), }\DataTypeTok{Libre =} \KeywordTok{sum}\NormalTok{(Libre), }\DataTypeTok{P_No_Cola=} \DecValTok{1}\OperatorTok{-}\KeywordTok{sum}\NormalTok{(Hizo_Cola)}\OperatorTok{/}\KeywordTok{sum}\NormalTok{(Cantidad),}
            \DataTypeTok{En_Cola=}\KeywordTok{sum}\NormalTok{(En_ColaS)}\OperatorTok{/}\KeywordTok{sum}\NormalTok{(Cantidad), }\DataTypeTok{En_Sistema=}\KeywordTok{sum}\NormalTok{(En_SistemaS)}\OperatorTok{/}\KeywordTok{sum}\NormalTok{(Cantidad)) }\OperatorTok\StringTok{ }
\StringTok{  }\KeywordTok{mutate}\NormalTok{(}\DataTypeTok{Por_Ocioso=}\NormalTok{ Libre}\OperatorTok{/}\NormalTok{Total_Minutos)}

\CommentTok{# Para poder calcular la distribucion por cada centro comercial}
\NormalTok{Totales <-}\StringTok{ }\NormalTok{datos_a_usar }\OperatorTok\StringTok{ }\KeywordTok{mutate}\NormalTok{(}\DataTypeTok{por_hora =}\NormalTok{ Cantidad}\OperatorTok{/}\NormalTok{Tiempo_Tot) }\OperatorTok\StringTok{ }\KeywordTok{group_by}\NormalTok{(CC, FinDe) }\OperatorTok\StringTok{ }
\StringTok{  }\KeywordTok{summarise}\NormalTok{(}\DataTypeTok{Total=}\KeywordTok{sum}\NormalTok{(por_hora))}

\NormalTok{datos_a_usar<-}\StringTok{ }\KeywordTok{as_data_frame}\NormalTok{(datos_a_usar) }\OperatorTok\StringTok{ }\KeywordTok{mutate}\NormalTok{(}\DataTypeTok{por_hora =}\NormalTok{ Cantidad}\OperatorTok{/}\NormalTok{Tiempo_Tot) }\OperatorTok\StringTok{ }\KeywordTok{full_join}\NormalTok{(Totales,}\DataTypeTok{by=}\KeywordTok{c}\NormalTok{(}\StringTok{"CC"}\NormalTok{, }\StringTok{"FinDe"}\NormalTok{)) }\OperatorTok\StringTok{ }
\StringTok{  }\KeywordTok{mutate}\NormalTok{(}\DataTypeTok{Distribucion=}\NormalTok{por_hora}\OperatorTok{/}\NormalTok{Total) }\OperatorTok\StringTok{ }
\StringTok{  }\CommentTok{# Reordenar columnas para mostrarlas ordenadas}
\StringTok{  }\KeywordTok{select}\NormalTok{(CC, Restaurante, FinDe, Lambda, Miu,  Distribucion, P_No_Cola, Por_Ocioso, T_Cola, T_Sistema_Prom, Max_T_Cola, En_Cola, En_Sistema, Sd_Miu, Min_T_Llegadas, Max_T_Llegadas, Inv_Miu, Inv_Lambda, Sd_Inv_Lambda, Cantidad, Tiempo_Tot)}
\end{Highlighting}
\end{Shaded}

\begin{verbatim}
## Warning: `as_data_frame()` is deprecated, use `as_tibble()` (but mind the new semantics).
## This warning is displayed once per session.
\end{verbatim}

\hypertarget{escribir-archivo}{%
\section{Escribir Archivo}\label{escribir-archivo}}

Esta linea de codigo hace que la tabla se cree y escribe un archivo csv
con estos datos. Para la visualizacion de este Markdown se mostró las
primeras filas de esta tabla

\begin{Shaded}
\begin{Highlighting}[]
\CommentTok{# Se escribe a un archivo donde se puede ver esta tabla a utilizar}
\KeywordTok{write.csv}\NormalTok{(datos_a_usar, }\DataTypeTok{file =} \StringTok{"Resultados/Datos Finales.csv"}\NormalTok{, }\DataTypeTok{row.names =} \OtherTok{FALSE}\NormalTok{)  }
\NormalTok{datos_a_usar[}\DecValTok{1}\OperatorTok{:}\DecValTok{13}\NormalTok{] }\OperatorTok\StringTok{ }\KeywordTok{head}\NormalTok{(}\DecValTok{15}\NormalTok{)}
\end{Highlighting}
\end{Shaded}

\begin{verbatim}
## # A tibble: 15 x 13
##    CC    Restaurante FinDe Lambda   Miu Distribucion P_No_Cola Por_Ocioso
##    <chr> <chr>       <dbl>  <dbl> <dbl>        <dbl>     <dbl>      <dbl>
##  1 La P~ BurgerKing      0  0.772 0.419       0.133     0.351    0.226   
##  2 La P~ BurgerKing      1  0.470 0.393       0.121     0.509    0.318   
##  3 La P~ GoGreen         0  0.472 0.257       0.0788    0.510    0.123   
##  4 La P~ GoGreen         1  0.176 0.208       0.0451    0.812    0.00582 
##  5 La P~ PandaExpre~     0  1.80  0.234       0.317     0.183    0       
##  6 La P~ PandaExpre~     1  0.805 0.237       0.206     0.0536   0.00204 
##  7 La P~ PolloCampe~     0  1.11  0.265       0.170     0.252    0.0599  
##  8 La P~ PolloCampe~     1  1.40  0.382       0.359     0.273    0.167   
##  9 La P~ Subway          0  0.715 0.277       0.125     0.0571   0.000568
## 10 La P~ Subway          1  0.508 0.222       0.122     0.211    0.0690  
## 11 La P~ TacoBell        0  1.00  0.292       0.176     0.128    0.0630  
## 12 La P~ TacoBell        1  0.573 0.231       0.147     0.344    0.00143 
## 13 Mira~ BurgerKing      0  0.284 0.328       0.0885    0.782    0.250   
## 14 Mira~ BurgerKing      1  0.766 0.533       0.159     0.242    0.312   
## 15 Mira~ GoGreen         0  0.230 0.158       0.0737    0.464    0.00287 
## # ... with 5 more variables: T_Cola <dbl>, T_Sistema_Prom <dbl>,
## #   Max_T_Cola <dbl>, En_Cola <dbl>, En_Sistema <dbl>
\end{verbatim}

\hypertarget{generar-una-lista-con-todas-las-simulaciones}{%
\section{Generar una lista con todas las
simulaciones}\label{generar-una-lista-con-todas-las-simulaciones}}

\begin{Shaded}
\begin{Highlighting}[]
\CommentTok{### El apply con Margin 1 significa que por cada fila del data frame correra la }
\CommentTok{###   función y se utiliza este n que es ingresado, en este caso son 12 horas.}
\NormalTok{Simulaciones <-}\StringTok{ }\KeywordTok{apply}\NormalTok{(datos_a_usar, }\DataTypeTok{MARGIN =} \DecValTok{1}\NormalTok{, }\DataTypeTok{FUN =}\NormalTok{ Generador_de_Cola, }\DataTypeTok{n=}\DecValTok{720}\NormalTok{)}
\end{Highlighting}
\end{Shaded}

\begin{verbatim}
## Warning: `as_data_frame()` is deprecated, use `as_tibble()` (but mind the new semantics).
## This warning is displayed once per session.
\end{verbatim}

\begin{Shaded}
\begin{Highlighting}[]
\KeywordTok{as_data_frame}\NormalTok{(Simulaciones[[}\DecValTok{1}\NormalTok{]]) }\OperatorTok\StringTok{ }\KeywordTok{head}\NormalTok{(}\DecValTok{20}\NormalTok{)}
\end{Highlighting}
\end{Shaded}

\begin{verbatim}
## # A tibble: 20 x 11
##    CC    Restaurante FinDe id    Entre_Llegadas Llegada Inicio  Cola   
##    <chr> <chr>       <chr> <chr> <Period>       <Perio> <Perio> <Perio>
##  1 La P~ BurgerKing  1     1     1M 1S          1M 1S   1M 1S   0S     
##  2 La P~ BurgerKing  1     2     1M 46S         2M 47S  2M 47S  0S     
##  3 La P~ BurgerKing  1     3     33S            3M 21S  3M 56S  34S    
##  4 La P~ BurgerKing  1     4     1M 51S         5M 12S  5M 12S  0S     
##  5 La P~ BurgerKing  1     5     3M 27S         8M 40S  9M 2S   22S    
##  6 La P~ BurgerKing  1     6     28S            9M 8S   9M 46S  38S    
##  7 La P~ BurgerKing  1     7     3M 24S         12M 33S 12M 33S 0S     
##  8 La P~ BurgerKing  1     8     23S            12M 56S 18M 14S 5M 17S 
##  9 La P~ BurgerKing  1     9     1M 4S          14M 0S  19M 7S  5M 7S  
## 10 La P~ BurgerKing  1     10    39S            14M 40S 21M 58S 7M 18S 
## 11 La P~ BurgerKing  1     11    1M 49S         16M 30S 25M 49S 9M 19S 
## 12 La P~ BurgerKing  1     12    40S            17M 10S 26M 37S 9M 27S 
## 13 La P~ BurgerKing  1     13    2M 29S         19M 39S 32M 0S  12M 20S
## 14 La P~ BurgerKing  1     14    42S            20M 22S 32M 27S 12M 5S 
## 15 La P~ BurgerKing  1     15    3S             20M 25S 36M 2S  15M 37S
## 16 La P~ BurgerKing  1     16    1S             20M 26S 36M 57S 16M 30S
## 17 La P~ BurgerKing  1     17    0S             20M 27S 37M 21S 16M 53S
## 18 La P~ BurgerKing  1     18    2M 16S         22M 43S 37M 46S 15M 3S 
## 19 La P~ BurgerKing  1     19    12S            22M 56S 44M 43S 21M 47S
## 20 La P~ BurgerKing  1     20    34S            23M 30S 47M 3S  23M 33S
## # ... with 3 more variables: Servicio <Period>, Final <Period>,
## #   En_Sistema <Period>
\end{verbatim}


\end{document}
